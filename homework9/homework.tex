\documentclass{article}

\usepackage{amsmath, amssymb, amsfonts}

\author{Matthew Dupraz}
\title{Exercice Bonus - Série 9}

\newcommand{\C}{\mathbb{C}}
\newcommand{\K}{\mathbb{K}}
\renewcommand{\L}{\mathbb{L}}

\begin{document}

\maketitle

Soit $f = x^7 - y^5 \in \C[x, y]$
Soit $\K = \C(y)$ et $\L$ le corps de décomposition de $f$ sur $\K$.
Soit $\alpha$ une racine de $f$ dans $\L$ et posons
$\beta = \frac{\alpha^3}{y^2}$. On va montrer que $f$ est
irréductible dans $\C[x, y]$

\subsection*{1.}

On va montrer que $[\K(\beta): \K] = 7$.
On a que $\alpha$ est une racine de $f$ et donc $\alpha^7 - y^5 = 0$
Il s'ensuit que $\frac{\alpha^7}{y^5} = 1$ et donc
$\frac{\beta^7}{y} = \frac{\alpha^{21}}{y^{15}} = 1$.
On déduit que $\beta^7 - y = 0$ et donc $\beta$ est une
racine du polynôme $g = x^7 - y \in \C[x, y]$.
Par le critère d'Eisenstein, puisque $g$ est primitif et 
$y$ est irréductible sur $\C[y]$, on a que $g$ est irréductible
sur $\C[y]$. Par le lemme de Gauss III, puisque $g$ est primitif,
$g$ est aussi irréductible sur $\K$. On a que
$g \in (m_{\beta, \K})$, mais alors puisque $g$ est irréductible
sur $\K$, on a forcément $g \sim m_{\beta, \K}$ et donc en
particulier, $\deg(m_{\beta, \K}) = 7$.
Ainsi $[\K(\beta): \K] = 7$.

\subsection*{2.}

On a que $\K(\beta) \subseteq \K(\alpha)$, puisque
$\beta = \frac{\alpha^3}{y^2} \in \K(\alpha)$.
Aussi, on a que $\K(\alpha) \subseteq \K(\beta)$, puisque
\begin{equation*}
\alpha = \alpha \frac{\alpha^{14}}{y^{10}}
= \frac{\alpha^{15}}{y^{10}} = \beta^5 \in \K(\beta)
\end{equation*}
Donc $\K(\alpha) = \K(\beta)$.

\subsection*{3.}

On a que
\begin{equation*}
[\K(\alpha):\K] = [\K(\alpha):\K(\beta)][\K(\beta):\K]
= 1\cdot 7 = 7
\end{equation*}
et donc
$\deg(m_{\alpha, \K}) = 7$. Puisque $f \in (m_{\alpha, \K})$
on a qu'il existe $h \in \K[x]$ tel que $f = h\cdot m_{\alpha, \K}$.
Puisque $\deg(f) = 7$, on déduit que $\deg(h) = 0$.
En particulier, $h \in \K = (\K[x])^\times$.
Ainsi $f \sim m_{\alpha, \K}$ et donc $f$ est irréductible sur
$\K$. Puisque $f$ est primitive, par le lemme de Gauss III on a
alors que $f$ est irréductible aussi sur $\C[y]$.

	
\end{document}
