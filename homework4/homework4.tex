\documentclass[french]{article}

\usepackage[utf8]{inputenc}
\usepackage{babel}
\usepackage[T1]{fontenc}

\usepackage{amsmath, amssymb, amsfonts}

\newcommand{\dx}{\frac{\partial}{\partial x}}
\newcommand{\mdx}[1]{\left(\frac{\partial}{\partial x}\right)^{#1}}
\newcommand{\DK}[1]{D_{\leq #1} (K[x])}
\newcommand{\N}{\mathbb{N}}
\newcommand{\lie}[2]{\left[#1, #2\right]}

\title{Série 4 - Exercice à rendre}
\author{Matthew Dupraz}

\begin{document}

\maketitle

\begin{enumerate}
	\item Soit $j \in \N$, alors
		\begin{equation*}
			\lie{\theta}{m_x}(x^{j-1}) = \theta(x^j) - x\theta(x^{j-1})\\
		\end{equation*}
		et donc
		\begin{align*}
			\theta(x^j) &= x\theta(x^{j-1}) + \lie{\theta}{m_x}(x^{j-1})\\
			&= x\theta(x^{j-1}) + \lie{\theta'}{m_x}(x^{j-1})\\
			&= x\theta(x^{j-1}) + \theta'(x^j) - x\theta'(x^{j - 1})\\
			&= \theta'(x^j) + x(\theta(x^{j-1}) - \theta'(x^{j-1}))
		\end{align*}
		On peut alors écrire en répétant le raisonnement pour $j-1$ etc.
		\begin{align*}
			\theta(x^j) - \theta'(x^j)
			&= x(\theta(x^{j-1}) - \theta'(x^{j-1}))\\
			&= x^2(\theta(x^{j-2}) - \theta'(x^{j-2}))\\
			&~\vdots\\
			&= x^j(\theta(x^0) - \theta'(x^0))\\
			&= x^j(\theta(1) - \theta'(1))\\
		\end{align*}
		Ainsi $(\theta - \theta')(x^j) = m_{\theta(1) - \theta'(1)}(x^j)$ pour
		tous $j \in \N$. Pour $j = 0$ c'est aussi (triviallement) vrai.
		Donc par linéarité des $\theta, \theta'$, on déduit que p.t. $p\in K[x]$
		$(\theta - \theta') = m_{\theta(1) - \theta'(1)}(p)$ et donc
		$\theta' \equiv \theta +  m_{\theta'(1) - \theta(1)}$.
	\item Soit $\theta \in D(K[x])$ t.q. $\lie{\theta}{m_x} = i\mdx{i-1}$,
		alors par l'exercice 6, $i\mdx{i-1} = \lie{\mdx{i}}{m_x}$ et donc par
		la partie 1, $\theta = \mdx{i} + m_\lambda$ pour un certain $\lambda \in
		K[x]$
	\item
		Posons
		\begin{equation*}
			E_s := \left\{\sum_{i=0}^s m_{p_i(x)}\mdx{i} \Big|~p_i(x) \in
			K[x]\right\}~~\forall s \geq 0
		\end{equation*}
		On va alors montrer que $\DK{s} = E_s$.
		
		Pour montrer que $E_s \subset \DK{s}$, on va faire une série de
		simplifications. Puisque
		$\DK{s}$ est stable par addition, et que
		$\DK{0} \subset \DK{1} \subset \dots$, il suffit
		de montrer que pour tout $p(x) \in K[x]$, $i\in \N_0$,
		\begin{equation*}
			m_{p(x)}\mdx{i} \in \DK{i}
		\end{equation*}
		De nouveau puisque $\DK{i}$ est stable par addition, et qu'on peut écrire
		$m_{p(x)}$ comme une combinaison linéaire des $m_{x^j}, j\in \N_0$, il
		suffit de montrer que p.t. $i, j\in \N_0$,
		\begin{equation*}
			m_{x^j}\mdx{i}\in \DK{i}
		\end{equation*}
		Pour montrer cela, il faut montrer que
		\begin{equation*}
			\lie{m_{x^j}\mdx{i}}{m_{q(x)}} \in \DK{i-1}
		\end{equation*}
		pour tout $q(x) \in K[x], i,j \in N_0$. Puisque le crochet de Lie est bilinéaire, et
		que $m_{q(x)}$ peut s'écrire comme une combinaison linéaire des $m_{x^t}$
		avec $t \in \N_0$, il suffit qu'on démontre que
		\begin{equation*}
			\lie{m_{x^j}\mdx{i}}{m_{x^t}} \in \DK{i - 1}
		\end{equation*}
		pour tous $i, j, t \in \N_0$. En faite, pour faire cela, on va démontrer
		une affirmation plus générale:
		\begin{equation*}
			\lie{m_{x^r}\mdx{i}m_{x^s}, m_{x^t}} \in \DK{i-1}
		\end{equation*}
		pour tous $i, r, s, t \in \N_0$.
		On va faire ceci par récurrence sur $i$:
		\begin{itemize}
			\item Si $i = 0$, alors on voit que
			\begin{equation*}
				\lie{m_{x^r}\mdx{i}m_{x^s}}{m_{x^t}} = \lie{m_{x^{r+s}}}{m_{x^{t}}} =
				m_0 \in \DK{-1}
			\end{equation*}
			pour tous $r, s, t \in \N_0$.
		\item Supposons que l'affirmation qu'on veut démontrer soit vraie pour un
			certain $i \in \N_0$, alors pour tous $r, s \in \N_0, t\in \N$
			\begin{align*}
				\lie{m_{x^r}\mdx{i}m_{x^s}}{m_{x^t}}
				&= m_{x^r}\mdx{i}m_{x^{s + t}} - m_{x^{r + t}}\mdx{i}m_{x^s}\\
				&= m_{x^r}\left(\mdx{i}m_{x^{t}} - m_{x^{t}}\mdx{i}\right)m_{x^s}\\
				&= m_{x^r}\lie{\mdx{i}}{m_{x^t}}m_{x^s}\\
				&= m_{x^r}\left(
					i\sum_{k=0}^{t - 1}m_{x^k}\mdx{i-1}m_{x^{t - 1 - k}}
				\right)m_{x^s}\\
				&= i\sum_{k=0}^{t - 1}m_{x^{k + r}}\mdx{i-1}m_{x^{t + s - 1 - k}}
			\end{align*}
			ce qui est par hypothèse de récurrence dans $\DK{i - 1}$.
			On a utilisé l'identité de l'exercice 6.2 dans ces égalités.
			Le cas où $t = 0$ est trivial, car le crochet s'annule.
		\end{itemize}
		Ceci démontre que 
		\begin{equation*}
			\lie{m_{x^r}\mdx{i}m_{x^s}, m_{x^t}} \in \DK{i-1}
		\end{equation*}
		pour tous $i, r, s, t \in \N_0$.
		Et donc par les raisonnements qu'on a fait, ceci démontre que $E_s \subset
		\DK{s}$ pour tous $s\geq 0$.

		Il faut maintenant montrer que $\DK{s} \subset E_s$.
		On va procédér par récurrence sur $s$.
		\begin{itemize}
			\item $\DK{0} = \{m_\lambda | \lambda \in K[x]\} = E_0$
			\item supposons que pour un certain $s \geq 0$,
				$\DK{s} \subset E_s$.
				Soit $\theta \in \DK{s + 1}$, alors
				$\lie{\theta}{m_x} \in \DK{s}$. Ainsi il existe
				$p_i(x) \in K[x]$ pour $i \in \{0, \dots, s\}$
				tels que $\lie{\theta}{m_x} =
				\sum_{i=0}^{s}m_{p_i(x)}\mdx{i}$
				On peut alors écrire
				\begin{align*}
					\lie{\theta}{m_x}
					&= \sum_{i=0}^{s}m_{p_i(x)}\mdx{i}\\
					&= \sum_{i=0}^{s}m_{(i+1)^{-1}(i+1)p_i(x)}\mdx{i}\\
				\end{align*}
				On effet, ici on a identifié
				\begin{equation*}
					i+1 = \underbrace{1_K + \dots 1_K}_{i+1\text{ fois}} \in K
				\end{equation*}
				ce qui est un élément dans $K$ et donc inversible. Ainsi on peut
				écrire
				\begin{align*}
					\lie{\theta}{m_x}
					&= \sum_{i=0}^{s}m_{(i+1)^{-1}p_i(x)}(i+1)\mdx{i}\\
					&= \sum_{i=0}^{s}m_{(i+1)^{-1}p_i(x)}\lie{\mdx{i+1}}{m_x}\\
					&= \sum_{i=0}^{s}m_{(i+1)^{-1}p_i(x)}
					\left(\mdx{i+1}m_x - m_x\mdx{i+1}\right)\\
					&= \sum_{i=0}^{s}\left(m_{(i+1)^{-1}p_i(x)}\mdx{i+1}m_x
					- m_{(i+1)^{-1}xp_i(x)}\mdx{i+1}\right)\\
					&= \sum_{i=0}^s \lie{m_{(i+1)^{-1}p_i(x)}\mdx{i+1}}{m_x}\\
					&= \sum_{i=1}^{s+1} \lie{m_{i^{-1}p_{i - 1}(x)}\mdx{i}}{m_x}\\
					&= \lie{\sum_{i=1}^{s+1}m_{i^{-1}p_{i - 1}(x)}\mdx{i}}{m_x}
				\end{align*}
				Par la partie 1. on a alors
				\begin{equation*}
					\theta = \sum_{i=1}^{s+1}m_{i^{-1}p_{i-1}(x)\mdx{i}}
					+ m_{\lambda}
				\end{equation*}
				pour un certain $\lambda \in K[x]$
				On en déduit que $\theta \in E_{s+1}$.
				Donc ceci démontre que $\DK{s+1} \subset E_{s+1}$
		\end{itemize}
		Donc on a bien que $\DK{s} \subset E_{s}$ pour tous $s\geq 0$,
		et puisqu'on a aussi montré l'autre inclusion, que
		$\DK{s} = E_s \forall s\geq 0$.

	\item
		Soit $I$ un idéal bilatère de $D(K[x])$, alors pour tous
		$\theta \in I, \phi \in D(K[x])$, on a que
		$\lie{\theta}{\phi} = \theta\phi - \phi\theta \in I$,
		car $I$ est preservé par multiplication à gauche et à droite.
		Supposons que $I$ contienne un élément non-nul $\theta$.
		Alors il exite $s \geq 0$ t.q. $\theta \in \DK{s}$.
		Ainsi par la partie 3., il existe $p_i(x)$ pour $i\in \{0, \dots, s\}$
		tels que $\theta = \sum_{i=0}^sm_{p_i(x)}\mdx{i}$.
		S.p.g. on peut supposer que $s$ soit minimal pour la propriété
		$p_s \neq 0$.
		On a alors
		\begin{align*}
			\lie{\theta}{m_x} &= \lie{\sum_{i=0}^sm_{p_i(x)}\mdx{i}}{m_x}\\
			&= \sum_{i=0}^s\lie{m_{p_i(x)}\mdx{i}}{m_x}\\
			&= \sum_{i=0}^s\left(m_{p_i(x)}\mdx{i}m_x - m_{xp_i(x)}\mdx{i}\right)\\
			&= \sum_{i=0}^sm_{p_i(x)}\lie{\mdx{i}}{m_x}\\
			&= \sum_{i=1}^sm_{p_i(x)}i\mdx{i-1}
			+ m_{p_0(x)}\lie{\mdx{0}}{m_x}\\
			&= \sum_{i=1}^sm_{p_i(x)}i\mdx{i-1}\\
			&= \sum_{i=0}^{s-1}m_{p_{i+1}(x)}(i+1)\mdx{i} \in \DK{s-1}\\
		\end{align*}
		Ainsi on voit que l'opération $\lie{\cdot}{m_x}$ diminue le degré de
		l'opérateur différentiel de 1. On voit que le premier terme de la somme a
		disparu et le dernier terme a été multiplié par le degré de l'opérateur.
		En répétant cette opération $s$ fois on
		obtient alors:
		\begin{equation*}
			\lie{\cdot}{m_x}^{s}(\theta) = m_{p_s(x)}s!\mdx{0} = s!m_{p_s(x)} \in I
		\end{equation*}
		Appelons $p(x) := s!p_s(x)$, alors on a que $m_{p(x)} \in I$.
		Par l'exercice de la semaine passé, 
		$\lie{\dx}{m_{p(x)}} = m_{\dx(p(x))}$. Comme le crochet de Lie préserve
		$I$, on a que $m_{\dx(p(x))} \in I$.
		En particulier, $m_{\mdx{t}(p(x))}\in I$, où $t = \deg(p(x))$.
		Mais si on pose $a = \mdx{t}(p(x))$, on a que $a$ est un polynôme de degré
		$0$, donc $a \in K$, et de plus comme $p(x)$ n'est pas nul, $a$ ne l'est
		pas non plus et donc $a \in K^\times$. Donc $a$ est inversible et pusique
		$I$ est preservé par les multiplications, $a^{-1}a =  1_K \in I$.
		En particulier, $D(K[x])\cdot 1_K = D(K[x]) \subset I$ et donc
		$I = D(K[x])$. Ceci démontre que les seuls idéaux bilatères de
		$D(K[x])$ sont l'idéal trivial $\{0_K\}$ et l'ensemble $D(K[x])$ lui-même.
		Donc $D(K[x])$ est simple.

\end{enumerate}

\end{document}
