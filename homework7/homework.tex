\documentclass{article}

\usepackage{amsmath, amssymb, amsfonts}

\newcommand{\Q}{\mathbb{Q}}
\newcommand{\C}{\mathbb{C}}
\newcommand{\Z}{\mathbb{Z}}
\newcommand{\ev}{\mathrm{ev}}
\newcommand{\pgcd}{\mathrm{pgcd}}

\title{Exercice bonus 7}
\author{Matthew Dupraz}

\begin{document}

\maketitle

\subsection*{1.}

On va montrer que $x^2 + y^2$ est irréductible dans
$\Q[x, y]$, mais pas dans $\C[x, y]$.

Soit $\ev_1: Q[x, y] \to Q[x]$ l'homomorphisme évaluation en $1$.
Pour montrer que $x^2 + y^2$ est irréductible dans $\Q[x, y]$,
il suffit par la Proposition 2.8.1 de montrer que
$\ev_1(x^2 + y^2) = x^2 + 1$ l'est dans $\Q[x]$.
Pour montrer cela, il suffit par la partie (d) de l'Exemple 2.3.7
de montrer que $c^2 + 1 \neq 0$ pour tous $c \in \Q$.
On a que $c^2 \geq 0$ et donc $c^2 + 1 \geq 1$ ce qui démontre le
résultat.

En revanche, $x^2 + y^2 = (x + iy)(x - iy)$ et donc $x^2 + y^2$
n'est pas irréductible dans $\C[x, y]$.

\subsection*{2.}

On va montrer que $x^3 - (y^7 + 2y^5 + y^3)$ est irréductible dans
$\Q[x, y]$.

De nouveau, par la Proposition 2.8.1, il suffit de montrer que
$\ev_1(x^3 - (y^7 + 2y^5 + y^3)) = x^3 - 4$ est irréductible dans
$\Q[x]$.
Par l'Exemple 2.3.7(d) il suffit donc de montrer qu'il n'existe pas
de $c \in \Q$ t.q. $c^3 - 4 = 0$. Supposons par l'absurde qu'il
existe un tel $c \in \Q$. Soient alors $p, q \in \Z$, tels que 
$q \neq 0$, $\pgcd(p, q) = 1$ et $c = \frac{p}{q}$.
Alors 
\begin{equation*}
	\left(\frac{p}{q}\right)^3 = 4
\end{equation*}
ce qui implique $p^3 = 4q^3$
En particulier, $p^3 \in (2)$ et puisque $2$ est
premier, $p \in (2)$. Ainsi $p = 2k$ pour un certain $k \in \Z$.
Alors on obtient que $(2k)^3 = 4q^3$ et donc
$8k^3 = 4q^3$. Mais ceci implique $2k^3 = q^3$ et donc
$q^3 \in (2)$. De nouveau, comme $2$ est premier, 
$q \in (2)$. Ceci contredit le fait que $\pgcd(p, q) = 1$, et
donc il n'existe pas de $c\in \Q$ t.q. $c^3 = 4$.
Ainsi $x^3 - 4$ est irréductible dans $\Q[x]$ et donc
$x^3 - (y^7 + 2y^5 + y^3)$ l'est aussi dans $\Q[x, y]$.


\end{document}
