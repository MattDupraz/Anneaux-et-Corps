\documentclass[french]{article}

\usepackage[utf8]{inputenc}
\usepackage{babel}
\usepackage[T1]{fontenc}

\usepackage{amsmath, amsfonts, amssymb}

\newcommand{\K}{\mathbb{K}}
\newcommand{\N}{\mathbb{N}}
\newcommand{\F}{\mathbb{F}}
\newcommand{\Z}{\mathbb{Z}}
\newcommand{\hF}{\hat{\F}}
\newcommand{\KK}{\K[[t]]\times(\K[[t]]\setminus\{0\})}
\newcommand{\KKL}{\K((t))\times(\K((t))\setminus\{0\})}
\newcommand{\ol}{\overline}

\title{Série 1 - Exercice à rendre}
\author{Matthew Dupraz}

\begin{document}

\maketitle

\subsection*{1.}

Soit $f(t) = \sum_{i=0}^\infty a_i t^i \in \K[[t]]$.

Supposons que $f$ soit inversible, alors il existe
$g(t) = \sum_{i=0}^\infty b_i t^i \in \K[[t]]$ tel que
$f(t)g(t) = 1$. Ainsi en particulier $f(0)g(0) = a_0 b_0 = 1$, 
ce qui implique $a_0 \neq 0$.

Sinon, supposons que $a_0 \neq 0$, alors on définit $b_k$ récursivement:
\begin{align*}
	b_0 &= a_0^{-1} \\
	b_k &= -a_0^{-1}\sum_{i=1}^{k}a_{k}b_{k - i}
\end{align*}
Alors on a que pour tout $k \in \N$, $\sum_{i=0}^k a_i b_{k-i} = 0$.
En particulier
\begin{equation*}
	\sum_{i=0}^\infty a_i t^i \sum_{j=0}^\infty b_j t^j
	= \sum_{k = 0}^\infty \sum_{i=0}^k a_i b_{k-i} = 1
\end{equation*}
Donc $f$ est inversible

\subsection*{2.}

$\K((t))$ est un anneau commutatif intègre, donc $\K[[t]] \subset \K((t))$
est aussi un anneau commutatif intègre.

Alors soit $\F$ le corps des fractions de $\K[[t]]$ et $\hat{\F}$ le corps des
fractions de $\K((t))$, on a alors que
$\F = \KK / \sim$  et
$\hF = \KKL / \sim$ où
$\sim$ est la relation d'équivalence telle que pour tout 
$(f(t), g(t)), (f'(t), g'(t)) \in \KKL$,
\begin{equation*}
	(f(t), g(t)) \sim (f'(t), g'(t)) \iff f(t)g'(t) = f'(t)g(t)
\end{equation*}

Soit $(f(t), g(t)) \in \KKL$ tel que $f(t) \neq 0$,
on va montrer qu'il existe 
$h(t) \in \K[[t]]$ et $k \in \Z$ tel que
$(f(t), g(t)) \sim (h(t), t^k)$ et $h(0) \neq 0$.
Soient 
$r = \min\{k \in \N, a_k \neq 0\}, s = \min\{k \in \N, b_k \neq 0\}$ où $a_k$ et
$b_k$ sont les coefficients de $t_k$ des séries $f(t)$ et $g(t)$.
$r, s$ existent forcément, car sinon $f(t)$ et $g(t)$ seraient nuls.
Ainsi on peut écrire $f(t) = p(t)t^r$ et $g(t) = q(t)t^s$ où
$p(0) \neq 0$ et $q(0) \neq 0$. Puisque $q(0) \neq 0$, $q(t)$ est inversible
et donc 
\begin{equation*}
	p(t)q^{-1}(t)g(t) = p(t)t^{s} = f(t)t^{s-r} 
\end{equation*}
Ainsi on a que $(f(t), g(t)) \sim (p(t)q^{-1}(t), t^{s-r})$ on remarque que 
$p(0)q^{-1}(0) \neq 0$. Ainsi si on pose $h(t) = p(t)q^{-1}(t)$ et $k = s-r$,
on a le résultat voulu.

Donc pour tout $\ol{(f(t), g(t))} \in \hF \setminus \{\ol{(0, 1)}\}$
il existe un $(h(t), t^k) \in \KKL$ tel que
$\ol{(f(t), g(t))} = \ol{(h(t), t^k)}$. On va montrer que cette représentation
est unique:

Soient $(f(t), t^r) \sim (g(t), t^s) \in \KKL$ tels que
$f(0) \neq 0$ et $g(0) \neq 0$,
alors $f(t)t^s = g(t)t^r$.
Si $r = s$, $f(t) = g(t)$ et donc $(f(t), t^r) = (g(t), t^s)$, sinon s.p.g.
supposons que $r > s$. On a alors que $f(t) = g(t)t^{r-s}$.
Si on évalue les deux côtés en zéro, on obtient $f(0) = 0$ ce qui est absurde.
Ainsi forcément $(f(t), t^r) = (g(t), r^s)$.

On pose alors $\hat{\phi}: \hF \to \K((t))$ définie par 
$\hat{\phi}(\ol{(h(t), t^k)}) = h(t)t^{-k}$ pour tous
$\ol{(h(t), t^k)} \in \hF\setminus\{\ol{(0, 1)}\}$ et
$\hat{\phi}(\ol{(0, 1)}) = 0$. On vérifie alors que $\hat{\phi}$ est un
homomorphisme d'anneaux. On a que $\hat\phi (\ol{(1, 1)}) = 1$ et
$\hat\phi (\ol{(0, 1)}) = 0$. 
Soeint $\ol{(f(t), t^r)}, \ol{(g(t), t^s)} \in \hF$,
Si $\ol{(f(t), t^r)} = \ol{(0, 1)}$, ou 
$\ol{(g(t), t^s)} = \ol{(0, 1)}$, alors les identités pour l'addition et
la multiplication sont triviallement vérifiées. Supposons alors que
$\ol{(f(t), t^r)} \neq \ol{(0, 1)}$ et 
$\ol{(g(t), t^s)} \neq \ol{(0, 1)}$, alors si on pose 
$l = \min \{r, s\}$ on a que
\begin{align*}
	\hat\phi(\ol{(f(t), t^r)} + \ol{(g(t), t^s)}) &=
	\hat\phi(\ol{(f(t)t^s + g(t)t^r, t^{r + s})})\\
	&= \hat\phi(\ol{(f(t)t^{s - l} + g(t)t^{r - l}, t^{r + s - l})})\\
	&= (f(t)t^{s-l} + g(t)^{r-l})t^{l - r - s} \\
	&= f(t)t^{-r} + g(t)t^{-s} \\
	&= \hat\phi(\ol{(f(t), t^r)}) + \hat\phi(\ol{(g(t), t^s)})
\end{align*}

Ainsi $\hat\phi$ préserve bien les additions. Finalement,
\begin{align*}
	\hat\phi(\ol{(f(t), t^r)} \cdot \ol{(g(t), t^s)}) &=
	\hat\phi(\ol{(f(t)g(t), t^{r+s})}) \\
	&= f(t)g(t)t^{-r-s}\\
	&= f(t)t^{-r}\cdot g(t)^{-s}\\
	&= \hat\phi(\ol{(f(t), t^r)})\cdot \hat\phi(\ol{(g(t), t^s)})	
\end{align*}
Donc $\hat\phi$ est bien un homomorphisme d'anneaux.

Soit $\iota: \F \to \hF$ l'homomorphisme inclusion, et posons
$\phi = \hat\phi \circ \iota$, alors $\phi$ est un homomorphisme d'anneaux. On
va montrer que $\phi$ est bijective:

Soient $\ol{(f(t), t^r)}$ et $\ol{(g(t), t^s)} \in \F$ t.q. 
$\phi(\ol{(f(t), t^r)}) = \phi(\ol{(g(t), t^s)})$. Alors
$f(t)t^{-r} = g(t)t^{-s}$, ce qui implique que $f(t)t^s = g(t)t^r$, mais alors
par définition de la relation d'équivalence,
$\ol{(f(t), t^r)} = \ol{(g(t), t^s)}$. Ainsi $\phi$ est injective.

Soit $\sum_{i = n}^\infty a_i t^i \in \K((t))$ une série de Laurent, alors
\begin{equation*}
\sum_{i = n}^\infty a_i t^i = t^{-n}\sum_{i = 0}^\infty a_{i - n} t^i
= \phi(\ol{(\sum_{i=0}^\infty a_{i-n}t^i, t^n)})
\end{equation*}
Ainsi $\phi$ est surjective et donc un isomorphisme.

On conclut que $\F \cong \K((t))$.

\end{document}
