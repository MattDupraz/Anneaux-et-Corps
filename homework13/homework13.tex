\documentclass[french]{article}

\usepackage[utf8]{inputenc}
\usepackage{amsmath, amssymb, amsfonts, amsthm}
\usepackage{tikz-cd}
\usepackage{float}

\newtheorem*{theorem}{Théorème}

\newcommand{\Q}{\mathbb{Q}}
\newcommand{\id}{\mathrm{id}}
\newcommand{\Gal}{\mathrm{Gal}}
\newcommand{\Aut}{\mathrm{Aut}}

\title{Exercice Bonus 13}
\author{Matt Dupraz}

\begin{document}

\maketitle

Soit $G$ un groupe fini. On va montrer que $G$ est un groupe
de Galois. On commence par un théorème:

\begin{theorem}[Cayley]
	Soit $G$ un groupe fini, alors il existe un monomorphisme
	$t: G \to S(G)$.
\end{theorem}
\begin{proof}
	Soit $g \in G$, on pose pour tout $x \in G$ 
	$t(g)(x) = gx$. On a que pour tous $x, x' \in G$,
	\begin{equation*}
		t(g)(x) = t(g)(x') \iff gx = gx' \iff x = x'
	\end{equation*}
	d'où $t(g)$ est une application injective, et puisque c'est une
	application ensembliste de $G$ dans $G$, elle est aussi
	bijective.
	Ainsi $t: G \to S(G)$ est bien définie et il reste à vérifier
	que c'est un monomorphisme.

	C'est un homomorphisme, car pour tous $g, h \in G$ et pour
	tous $x \in G$, $t(gh)(x) = ghx = t(g)(hx)
	= (t(g)\circ t(h))(x)$. Finalement, $t$ est injective, car
	pour tout $g \in G$,
	\begin{equation*}
		t(g) = \id_G \iff t(g)(x) = x ~\forall x \in G
			\iff gx = x ~\forall x \in G \iff g = e
	\end{equation*}
	et donc $\ker(t) = \{e\}$.
\end{proof}

Posons $n = |G|$, alors par le theorème de Cayley,
$G$ est isomorphe à un sous-groupe de $S(G) \cong S_n$.
Ainsi on peut 
identifier les éléments de $G$ avec des permutations dans $S_n$.
Soit $A := \Q[x_1, \dots, x_n]$.
On définit alors pour tout $\sigma \in G$, l'automorphisme 
$\phi_\sigma : A \to A$
par $\phi_\sigma(x_i) = x_{\sigma(i)}$ pour tout
$i \in \{1, \dots, n\}$. Il est facile à voir que c'est un
homomorphisme d'anneaux et que son inverse est donné par
$\phi_{\sigma^{-1}}$.
Soit $K$ le corps de fraction de $A$, alors par
la propriété universelle du corps de fractions, pour tout
$\sigma \in G$, il existe un unique homomorphisme de corps
$\psi_\sigma: K \to K$
qui fait commuter
le diagramme suivant:
\begin{equation*}
\begin{tikzcd}
A \arrow[r, "\phi_\sigma"] \arrow[d, hook] & A \arrow[d, hook] \\
K \arrow[r, "\psi_\sigma", dashed]         & K                
\end{tikzcd}
\end{equation*}

On définit $H = \{\psi_\sigma: \sigma \in G\}$. On va montrer
que $H$ est un groupe pour la composition. Soit $e \in G$
l'élement neutre, alors on a que $\phi_e = \id_A$, et donc le
diagramme suivant commute:
\begin{equation*}
\begin{tikzcd}
	A \arrow[r, "\id_A"] \arrow[d, hook] & A \arrow[d, hook] \\
K \arrow[r, "\psi_{e}"]         & K
\end{tikzcd}
\end{equation*}
Ainsi par la partie unicité de la propriété universelle du
corps de fractions, $\psi_e = \id_K$, et donc $H$ admet un
élément neutre.

Soient $\sigma, \tau \in G$, alors le diagramme suivant commute:
\begin{equation*}
\begin{tikzcd}
A \arrow[r, "\phi_\sigma"] \arrow[d, hook] & A \arrow[d, hook] \arrow[r,"\phi_{\tau}"] & A \arrow[d, hook] \\
K \arrow[r, "\psi_\sigma"]                 & K \arrow[r, "\psi_{\tau}"]                 & K
\end{tikzcd}
\end{equation*}
De plus, on a que
$\phi_{\tau}\circ\phi_{\sigma} = \phi_{\tau\sigma}$, et donc
le diagramme suivant commute:
\begin{equation*}
\begin{tikzcd}
	A \arrow[r, "\phi_{\tau\sigma}"] \arrow[d, hook] & A \arrow[d, hook] \\
K \arrow[r, "\psi_{\tau}\circ\psi_\sigma"]         & K
\end{tikzcd}
\end{equation*}
On conclut par la partie unicité de la
propriété universelle du corps de fractions que
$\psi_\tau\circ\psi_\sigma = \psi_{\tau\sigma} \in H$.
Ainsi $H$ est stable sous composition et la composition est
clairement associative.

Finalement, pour tout $\sigma \in G$, on a que
$\psi_\sigma \circ \psi_{\sigma^{-1}} = 
\psi_{\sigma\sigma^{-1}} = \psi_e = \id_K$,
et de même $\psi_{\sigma^{-1}} \circ \psi_\sigma = \id_K$.
On a montré alors que $H$ est un groupe et, en particulier,
$H \leq \Aut(K)$.

Pour conclure, on a que pour tout $\psi_\sigma \in H$,
$\psi_\sigma|_\Q = \phi_\sigma|_\Q = \id_\Q$, ainsi
$H \leq \Aut_\Q(K) = \Gal(K/\Q)$.
Donc, par le theorème fondamental de la théorie de Galois,
$K^H \subseteq K$ est une extension galoisienne telle que 
$\Gal(K/K^H) = H \cong G$. On conclut que $G$ est un groupe de
Galois.

\end{document}
