\documentclass{article}

\usepackage{amsmath, amsfonts, amssymb}

\newcommand{\F}{\mathbb{F}}

\author{Matthew Dupraz}
\title{Série 10 - Exercice bonus}

\begin{document}

\maketitle

\subsection*{1.}

Soif $f \in \F_p[x]$ un polynôme unitaire irréductible de
degré $d$. Par l'exercice 5, $f$ n'a pas de racines multiples
dans $\F_{p^d}$.
De plus si $g \in \F_p[x]$ est un autre polynôme irréductible
de degré $d$, alors puisque $f$, et $g$ sont unitaires, ils ne
peuvent pas être associés. On a alors par l'exercice 5 que
$f$ et $g$ n'ont aucune racine en commun. Notons $E$ l'ensemble
de toutes les racines des polynômes unitaires irréductibles
sur $\F_p$ de degré $d$.
Alors le raisonnement précédent implique que
$|E| = d\cdot N_d$, où $N_d$ est le nombre de polynômes unitaires
irréductibles sur $\F_p$ de degré $d$.

Pour conclure, on va démontrer que $E = \F_{p^d} \setminus
\bigcup_{L \subsetneq \F_{p^d}}L$.
Soit $\alpha \in E$, alors $\alpha$ est une racine d'un unique
polynôme $f$ unitaire irréductible sur $\F_p$ de degré $d$.
En particulier, $f \sim m_{\alpha, \F_p}$ car $f$ est
irréductible. Ceci implique que 
$[\F_p(\alpha):\F_p] = d$ et puisque
$[\F_{p^d} : \F_p] = d$, on a forcément
$[F_{p^d}:\F_{p}(\alpha)] = 1$,
et donc $\F_p(\alpha) = \F_{p^d}$.
Ainsi $\alpha$ n'appartient à aucun sous-corps propre de $\F_{p^d}$,
d'où $\alpha \in \F_{p^d} \setminus
\bigcup_{L \subsetneq \F_{p^d}}L$. 

Pour l'autre inclusion, soit $\alpha \in \F_{p^d}\setminus
\bigcup_{L \subsetneq \F_{p^d}}L$.
Soit $m_{\alpha, \F_p} \in \F_p[x]$,
alors forcément $\deg m_{\alpha, \F_p} = d$,
car sinon $[\F_p(\alpha): \F_p] < d$ et donc $\alpha$ appartiendrait
à un sous-corps propre de $\F_{p^d}$. Mais alors si $c \in F_p$ est
le coefficient dominant de $m_\alpha$, on a que
$c^{-1}m_{\alpha, \F_p}$ est un polynôme unitaire irréductible
sur $\F_p$ de degré $d$ dont $\alpha$ est une racine.
Ainsi $\alpha \in E$. On conclut alors que $E = \F_{p^d}\setminus
\bigcup_{L \subsetneq \F_{p^d}}L$.

Ceci démontre que
\begin{equation*}	
	d\cdot N_d = |E| = |\F_{p^d}\setminus
	\bigcup_{L \subsetneq \F_{p^d}}L|
\end{equation*}

\subsection*{2.}

En appliquant le Corollaire 3.4.22 on peut retrouver tous les
sous-corps de $F_{p^d}$ et donc calculer $|\F_{p^d}\setminus
\bigcup_{L \subsetneq \F_{p^d}}L|$. Pour chaque $r|s$,
dans tout ce qui suit on
va identifier l'unique sous-corps de $\F_{p^s}$ qui est isomorphe
à $\F_{p^r}$ avec $\F_{p^r}$.

\begin{itemize}
	\item On a que $\F_{p}$ est l'unique sous-corps propre
		de $\F_{p^2}$. Donc
		\begin{equation*}
			|\F_{p^2}\setminus
			\bigcup_{L \subsetneq \F_{p^2}}L|
			= |\F_{p^2} \setminus \F_p| = p^2 - p
		\end{equation*}
		Ainsi on obtient
		\begin{equation*}
			N_2 = \frac{p^2 - p}{2}
		\end{equation*}
	\item On a que $\F_{p}$ est l'unique sous-corps propre
		de $\F_{p^3}$. Donc
		\begin{equation*}
			|\F_{p^3}\setminus
			\bigcup_{L \subsetneq \F_{p^3}}L| =
			|\F_{p^3} \setminus \F_p| = p^3 - p
		\end{equation*}
		Ainsi on obtient
		\begin{equation*}
			N_3 = \frac{p^3 - p}{3}
		\end{equation*}
	\item On a que $\F_{p^2}$ et $\F_{p}$ sont les uniques
		sous-corps propres de $\F_{p^4}$,
		mais $\F_p \subseteq \F_{p^2}$, donc 
		\begin{equation*}
			|\F_{p^4}\setminus
			\bigcup_{L \subsetneq \F_{p^4}}L| =
			|\F_{p^4} \setminus \F_{p^2}| = p^4 - p^2
		\end{equation*}
		Ainsi on obtient
		\begin{equation*}
			N_4 = \frac{p^4 - p^2}{4}
		\end{equation*}
	\item On a que $\F_{p}$ est l'unique sous-corps propre
		de $\F_{p^5}$	
		\begin{equation*}
			|\F_{p^5}\setminus
			\bigcup_{L \subsetneq \F_{p^5}}L| =
			|\F_{p^5} \setminus \F_p| = p^5 - p
		\end{equation*}
		Ainsi on obtient
		\begin{equation*}
			N_5 = \frac{p^5 - p}{5}
		\end{equation*}
	\item On a que $\F_{p}, \F_{p^2}, \F_{p^3}$ sont tous les
		sous-corps de $\F_{p^6}$. On a que 
		$\F_{p} \subseteq \F_{p^2}$ et $\F_{p} \subseteq \F_{p^3}$,
		ainsi forcément $\F_{p^2}\cap\F_{p^3} = \F_p$.
		Donc par le principe d'inclusion exclusion on obtient
		\begin{align*}
			|\F_{p^6}\setminus
			\bigcup_{L \subsetneq \F_{p^6}}L| &=
			|\F_{p^6} \setminus (\F_{p^2} \cup \F_{p^3})|\\
			&= |\F_{p^6}| - (|\F_{p^2}| + |\F_{p^3}| - |\F_{p}|)\\
			&= p^6 - p^3 - p^2 + p
		\end{align*}
		Ainsi on obtient
		\begin{equation*}
			N_6 = \frac{p^6 - p^3 - p^2 + p}{6}
		\end{equation*}
\end{itemize}

\subsection*{3.}

Supposons que $n,m$ sont premiers entre eux,
alors on va montrer que 
$\F_{p^{d/n}} \cap \F_{p^{d/m}} = \F_{p^{d/nm}}$.
On a que $\F_{p^{d/n}} \cap \F_{p^{d/m}} = \F_{p^r}$ pour un
certain $r$ qui divise $d/n$ et $d/m$. Donc il existe
$a, b \in \mathbb{N}$ tels que $arn = d = brm$. En particulier
$an = bm$ et donc puisque $n,m$ sont premiers entre eux,
$a | m$ et donc $a = cm$ pour un certain $c\in \mathbb{N}$.
Ainsi $cr = d/nm$ d'où $r$ divise $d/nm$.
Mais puisque $\F_{p^{d/nm}} \subseteq \F_{p^{d/n}}
\cap \F_{p^{d/m}} = \F_{p^r}$, on a que $d/nm$ divise $r$.
On conclut que $r = d/nm$

\subsection*{4.}

Soit $d = s_1^{i_1} \dots s_n^{i_n}$ la décomposition de $d$
en facteurs premiers.
On a que
\begin{equation*}
	\bigcup_{L \subsetneq \F_{p^d}}L
	= \bigcup_{\substack{k|d \\ k \neq d}}\F_{p^k}
	= \bigcup_{j = 1}^n\F_{p^{d/s_j}}	
\end{equation*}
car si $k|d$ et $k \neq d$, alors $k|d/s_j$ pour un ceratin $j$, et
alors $\F_{p^k} \subseteq \F_{p^{d/s_j}}$.
Par le principe d'inclusion-exclusion, on a que
\begin{align*}
	|\bigcup_{j = 1}^n\F_{p^{d/s_j}}|
	&= \sum_{k = 1}^n (-1)^{k+1}\left(
		\sum_{1 \leq j_1 < \dots < j_k \leq n}
		|\F_{p^{d/s_{j_1}}} \cap \dots \cap \F_{p^{d/s_{j_k}}}|
	\right) \\
	&= \sum_{k = 1}^n (-1)^{k+1}\left(
		\sum_{1 \leq j_1 < \dots < j_k \leq n}
		|\F_{p^{d/s_{j_1}\cdots s_{j_k}}}|
	\right) \\ 
	&= \sum_{k = 1}^n (-1)^{k+1}\left(
		\sum_{1 \leq j_1 < \dots < j_k \leq n}
		p^{d/s_{j_1}\cdots s_{j_k}}
	\right)
\end{align*}
On a alors que
\begin{align*}
	|\F_{p^d} \setminus \bigcup_{L \subsetneq \F_{p^d}}L|
	&= p^d + \sum_{k = 1}^n (-1)^{k}\left(
		\sum_{1 \leq j_1 < \dots < j_k \leq n}
		p^{d/s_{j_1}\cdots s_{j_k}}
	\right)\\
	&= p^d + \sum_{k = 1}^n \left(
		\sum_{1 \leq j_1 < \dots < j_k \leq n}
		(-1)^k p^{d/s_{j_1}\cdots s_{j_k}}
	\right)\\
	&= \mu(1)p^d + \sum_{k = 1}^n \left(
		\sum_{1 \leq j_1 < \dots < j_k \leq n}
		\mu(s_{j_1}\cdots s_{j_k}) p^{d/s_{j_1}\cdots s_{j_k}}
	\right)\\
	&= 
	\sum_{\substack{m_1, \dots, m_n \\ 0 \leq m_k \leq i_k}}
	\mu(s_1^{m_1}\cdots s_n^{m_n}) p^{d/s_1^{m_1}\cdots s_n^{m_n}}\\
	&= \sum_{r|d} \mu\left( \frac{d}{r}\right)p^r
\end{align*}
À la quatrième égalité on a rajouté des termes à la somme, mais on
remarque que si $m_k \geq 2$ pour un certain $k$, alors 
$\mu(s_1^{m_1} \cdots s_n^{m_n}) = 0$, ainsi tous les termes
qu'on rajouté à la somme sont en fait nuls.
On conclut que
\begin{equation*}
	N_d = \frac{1}{d}
	|\F_{p^d} \setminus \bigcup_{L \subsetneq \F_{p^d}}L|
	= \frac{1}{d}\sum_{r|d} \mu\left( \frac{d}{r}\right)p^r
\end{equation*}

\end{document}
