\documentclass{article}

\usepackage{amsmath, amssymb, amsfonts}

\newcommand{\F}{\mathbb{F}}
\newcommand{\K}{\mathbb{K}}
\newcommand{\N}{\mathbb{N}}
\newcommand{\sef}{\F[[t]]}
\newcommand{\sel}{\F((t))}
\newcommand{\ord}{\mathrm{ord}}
\newcommand{\ev}{\mathrm{ev}}

\author{Matthew Dupraz}
\title{Exercice bonus - Série 8}

\begin{document}

\maketitle

\subsection*{(a)}

Soit $\F$ un corps, on va montrer que $\sef$ est un
anneau factoriel.

Soit $f = \sum_{i=0}^\infty a_it^i \in \sef$, alors on définit
\begin{equation*}
\ord(f) = \begin{cases}
	\min\{k\in \N: a_k \neq 0\} & \text{ si } f \neq 0\\
	-\infty & \text{ si } f = 0
\end{cases}
\end{equation*}

Si $f \neq 0$, on peut alors
écrire $f = t^{\ord(f)}g$ où le coefficient de $t^0$ de $g$ est
non-nul. En particulier, ceci implique par l'exercice 9.1 de la
série 1 que $g$ est inversible dans $\sef$.

On va montrer que 
$\ord$ est une fonction euclidienne. Soient $f \in \sef\setminus
\{0\}$ et $g \in \sef$. Alors si $\ord(g) < \ord(f)$, on peut
simplement écrire $g = 0\cdot f + g$, supposons alors que
$\ord(g) \geq \ord(f)$. On a alors que $f = t^{\ord(f)}a$ et
$g = t^{\ord(g)}b$ pour $a, b \in \sef^\times$.
En particulier,
\begin{equation*}
	g = t^{\ord(g)}a = t^{\ord(g) - \ord(f)}ab^{-1}\cdot f + 0
\end{equation*}
On a que $\ord(t^{\ord(g) - \ord(f)}ab^{-1}) = \ord(g) - \ord(f)
< \ord (g)$ et donc ceci démontre que $\ord$ est bien une fonction
euclidienne.

Donc $\sef$ est un anneau euclidien, ce qui implique qu'il est
principal et donc factoriel.

\subsection*{(b)}

Avant de continuer, on vérifie une propriété de $\ord$ qui nous
sera utile -- pour tous $f, g \in \sef$, $\ord(f\cdot g) = 
\ord(f) + \ord(g)$. On a que $f = t^{\ord(f)}a$ et
$g = t^{\ord(g)}b$ où $a, b \in \sef^\times$. Ainsi
$f\cdot g = t^{\ord(f) + \ord(g)}ab$. On a que $ab \in \sef^\times$
et donc le coefficient de $t^0$ est non-nul. Il s'ensuit alors
que forcément $\ord(f\cdot g) = \ord(f) + \ord(g)$ et donc
$\ord$ est bien multiplicative.

On va montrer que $x^{2021} - t^{42} \in (\sef)[x]$ est
irréductible. On a vu dans l'exercice 9 de la série 1 que le
corps de fractions de $\sef$ c'est $\sel$.
Le Lemme $3.3.3$ assure l'existence d'un corps de décomposition de
$x^{2021} - t^{42}$ sur $\sel$, notons le $\K$. On a alors qu'il
existe $c, \alpha_1, \dots, \alpha_{2021} \in \K$ tels que
\begin{equation*}
	x^{2021} - t^{42} = c\prod_{i = 1}^{2021}(x - \alpha_i)
\end{equation*}
Puisque le coefficient dominant de $x^{2021} - t^{42}$ c'est $1$,
on déduit que $c = 1$

Raisonnons par l'absurde et supposons
que $x^{2021} - t^{42}$ ne soit pas
irréductible sur $\sef$, alors il existe $f, g \in (\sef)[x]$
non-inversibles tels que $x^{2021} - t^{42} = f\cdot g$.
Puisque $K[x]$ est un anneau factoriel, par
l'unicité de décomposition dans un anneau factoriel, on a qu'il
existe une partition $I \sqcup J = \{1, \dots, 2021\}$ et
$u \in \K$ tels que
\begin{equation*}
	f = u\prod_{i \in I} (x - \alpha_i)
	~~\text{ et }~~
	g = u^{-1}\prod_{j \in J} (x - \alpha_j)
\end{equation*}
Par supposition, $f$ et $g$ ne sont pas inversibles et donc $I, J$
sont non-vides.
On a alors que $\alpha_I := \prod_{i\in I}\alpha_i \in \sef$.
Puisque $x - \alpha_i|x^{2021} - t^{42}$, on a que 
$\alpha_i^{2021} - t^{42} = 0$ pour tous $i \in I$.
En particulier, 
\begin{equation*}
	\alpha_I^{2021} = \left(\prod_{i\in I}\alpha_i\right)^{2021}
	= \prod_{i\in I}\alpha_i^{2021}
	= \prod_{i \in I}t^{42} = t^{42|I|}
\end{equation*}
Il s'ensuit que $\ord(\alpha_I^{2021}) = \ord(t^{42|I|})$ ce qui
implique $2021\cdot\ord(\alpha_I) = 42|I|$ par la multiplicativité
de $\ord$. Puisque $2021 = 43\cdot47$, on a que $2021$ et $42$
sont premiers entre eux et donc forcément que
$2021$ divise $|I|$. Comme $|I| \leq 2021$, ceci implque que
$|I| = 2021$, ce qui est absurde, car alors $|J| = 0$.

Ceci démontre alors que $x^{2021} - t^{42}$ est irréductible
sur $\sef$.

\subsection*{(c)}

On va montrer que $x^{2021} + y^{2021} - t^{42}$ est irréductible
dans $(\sef)[x, y]$. Soit $\ev_0: (\sef)[y][x] \to (\sef)[x]$
l'homomorphisme évaluation en 0, alors
\begin{equation*}
	\ev_0(x^{2021} + y^{2021} - t^{42}) = x^{2021} - t^{42}
\end{equation*}
ce qui est irréductible par la partie (b). On conclut par la
Proposition 2.8.1 que $x^{2021} + y^{2021} - t^{42}$ est
irréductible dans $(\sef)[x, y]$.

\end{document}
