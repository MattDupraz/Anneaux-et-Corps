\documentclass{article}

\usepackage{amsmath, amsfonts, amssymb}
\usepackage[utf8]{inputenc}

\newcommand{\F}{\mathbb{F}}

\author{Matthew Dupraz}
\title{Exercice à rendre - Série 11}

\begin{document}

\maketitle

\subsection*{(a)}

Soit $K$ un corps de caractéristique $p > 0$,
soit $\alpha \in K \setminus K^p$. On va montrer que
$x^p - \alpha \in K[x]$ est irréductible.

Soit $M$ le corps de décomposition de $x^p - \alpha$,
alors il existe $\beta \in M$ tel que $\beta^p = \alpha$.
Alors $x^p - \alpha = (x - \beta)^p$, car on est dans un corps de
caractéristique $p$. On a que $(x - \beta)^k \notin K[x]$ pour tout
$k \in \{1, \dots, p-1\}$, car le coefficient de $x^{k-1}$ c'est
$-\beta\binom{k}{k-1} = - k\beta \notin K$ ($p$ ne divise pas $k$).
Ainsi $x^p - \alpha$ est
forcément irréductible sur $K$.

\subsection*{(b)}

Pour montrer que $L$ est un corps, il faut montrer que
$(y^2 - x(x-1)(x+1))$ est maximal, i.e. que $y^2 - x(x-1)(x+1)$ est irréductible
sur $\F_p(x)$. Puisque $\F_p[x]$ est un anneau factoriel et
$y^2 - x(x-1)(x+1)$ est un polynôme primitif, par le lemme de Gauss III, 
on a que $y^2 - x(x-1)(x+1)$ est irréductible si et seulement si il l'est
sur $\F_p[x]$. Puisque c'est un polynôme de degré 2, ceci équivaut à dire
qu'il ne s'annule pour aucun $f \in \F_p[x]$.
On a que pour tous $f \in \F_p[x]$,
$\deg(f^2) = 2\deg(f) \neq 3$ et donc $f^2 \neq x(x-1)(x+1)$.
On conclut que $y^2 - x(x-1)(x+1)$ ne s'annule pour aucun $f \in \F_p[x]$ et
donc $L$ est bien un corps.

\subsection*{(c)}

Pour commencer, on démontre que $x \in \F_p(x) \setminus \F_p(x)^p$.
Pour tous $\frac{f}{g} \in \F_p(x)$ on peut définir
$\deg(\frac{f}{g}) = \deg(f) - \deg(g)$. Il est facile à voir que c'est bien
défini et multiplicatif. Alors pour tout $\frac{f^p}{g^p} \in \F_p(x)^p$, 
on a forcément que $p | \deg{f^p}{g^p} = p\deg(f) - p\deg(g)$.
Mais $\deg(x) = 1$ et donc forcément $x \notin \F_p(x)^p$.

Raisonnons par l'absurde et supposons que $x \in L^p$, alors soit $a \in L^p$
tel que $a^p = x$. On a alors la chaîne d'inclusions
$\F_p(x) \subseteq \F_p(x)(a) \subseteq L$.
Par la partie $(a)$, le polynôme $y^p - x$ est irréductible sur $\F_p(x)$, on 
en déduit que $m_{a, \F_p(x)} \sim y^p - x$. On en déduit alors que
$[\F_p(x)(a) : \F_p(x)] = p$, mais on a que $[L : \F_p(x)] = 2$, donc ceci est
absurde, car $[L:\F_p(x)] = [L:\F_p(x)(a)][\F_p(x)(a):\F_p(x)]$ et $p > 2$.
Ainsi $x \in L\setminus L^p$ et donc on conclut par la proposition 3.5.8 que
$L$ n'est pas parfait.

\subsection*{(d)}

On démontre maintenant le cas $p = 2$.
Pour cela, on va démontrer que $L = \F_2(\sqrt{x})$
On a que
\begin{equation*}
	y^2 - x(x+1)(x-1) = y^2 + x(x+1)^2 = (y + \sqrt{x}(x+1))^2
\end{equation*}
et donc
$L = \F_2(x)(\sqrt{x}(x+1)) = \F_2(x)(\sqrt{x})$, car
$\sqrt{x}(x+1) \in \F_2(x)(\sqrt{x})$ et
$\sqrt{x} = \frac{\sqrt{x}(x+1)}{x+1} \in \F_2(x)(\sqrt{x}(x+1))$.
Aussi, $\F_2(x)(\sqrt{x}) = \F_2(\sqrt{x})$, car 
$\sqrt{x} \in \F_2(x)(\sqrt{x})$ et $x, \sqrt{x} \in \F_2(\sqrt{x})$.
On conclut que $L = \F_2(\sqrt{x})$.
Il s'ensuit que $L^2 = \F_2(\sqrt{x})^2 = \F_2^2(x) = \F_2(x)$,
car pour tous
$\sum a_i\sqrt{x}^i / \sum b_j\sqrt{x}^j \in \F_2(\sqrt{x})$,
on a que
\begin{equation*}
	\left(\frac{\sum a_i\sqrt{x}^i}{\sum b_i\sqrt{x}^j}\right)^2
	= \frac{\left(\sum a_i\sqrt{x}^i\right)^2}
	{\left(\sum b_i\sqrt{x}^j\right)^2}
	= \frac{\sum a_i^2 \sqrt{x}^{2i}}{\sum b_j^2\sqrt{x}^{2j}}
	= \frac{\sum a_i^2 x^i}{\sum b_j^2x^j}
\end{equation*}
On conclut que $L \neq L^2$ et donc $L$
n'est pas parfait par la proposition $3.5.8$.

\end{document}
